\documentclass{article}
\usepackage{amsmath}
\usepackage{amssymb}
\usepackage{amsthm}

\newcommand{\modx}{ \ \text{mod} \ }
\newcommand{\E}{\mathbb{E}}
\newcommand{\Prob}{\mathbb{P}}

\newtheorem{theorem}{Theorem}
\newtheorem{lemma}{Lemma}
\newtheorem{definition}{Definition}

\title{Approach 1: Upper Bound Method}
\author{}
\date{}

\begin{document}
\maketitle

\section{Goal}
Compute an upper bound on $\E[\text{\# of laps}]$ for any strategy, then show our wait-time strategy achieves it.

\section{Setup}
\begin{itemize}
    \item Particle has velocity $v > 0$, lap time $\tau = 2/v$
    \item Lifespan $l \sim \text{Exp}(\lambda)$, i.e., $\Prob(l > t) = e^{-\lambda t}$
    \item Gate ON during $(0, T_1]$, OFF during $(T_1, T_1+T_2]$, period $T = T_1 + T_2$
    \item Strategy: function $w(l, \Omega)$ giving wait time based on lifespan and current offset
\end{itemize}

\section{Attempt 1: Naive Upper Bound}

The simplest upper bound ignores the gate entirely. Without a gate, survival probability per lap is:
\[
S_0 = \Prob(l \geq \tau) = e^{-\lambda \tau}
\]
giving expected laps:
\[
\E[\text{\# of laps}] \leq \frac{S_0}{1 - S_0} = \frac{e^{-\lambda \tau}}{1 - e^{-\lambda \tau}} = \frac{1}{e^{\lambda \tau} - 1}
\]

This bound is too loose --- it doesn't account for the gate at all.

\section{Attempt 2: Gate-Aware Upper Bound}

For a particle to complete a lap, it must:
\begin{enumerate}
    \item Have lifespan $l \geq w + \tau$ (survive the wait plus travel)
    \item Arrive when gate is ON
\end{enumerate}

\textbf{Key observation:} The best possible scenario is when the particle can always reach an ON window with zero wait. This happens when $\Omega + \tau \leq T_1$ (mod $T$), i.e., the particle naturally arrives during ON.

But even in the best case, the particle needs $l \geq \tau$. So:
\[
\E[\text{\# of laps}] \leq \frac{e^{-\lambda \tau}}{1 - e^{-\lambda \tau}}
\]

This is the same bound! The gate can only hurt, never help.

\section{Attempt 3: Tighter Bound via Conditional Analysis}

Let's condition on the offset $\Omega$ at the start of a lap. Define:
\[
S(\Omega) = \Prob(\text{survive lap} \mid \text{start at offset } \Omega)
\]

For a particle starting at offset $\Omega \in (0, T_1]$:
\begin{itemize}
    \item It arrives at time $\Omega + w + \tau$ (mod $T$)
    \item To arrive in ON window $(0, T_1]$, need $\Omega + w + \tau \in (0, T_1] \pmod{T}$
    \item Also need $l \geq w + \tau$
\end{itemize}

The optimal $w$ maximizes survival probability. But here's the key insight: the particle knows $l$ before choosing $w$. So it can always choose $w$ to land in an ON window if possible.

\textbf{When is it possible?} The particle can reach the $k$-th ON window if:
\[
l \geq w_k + \tau
\]
where $w_k$ is the wait needed to reach that window. The latest reachable ON window gives the optimal positioning.

\section{Difficulty with Upper Bound Approach}

The challenge is that the upper bound approach doesn't naturally capture the \emph{adaptive} nature of the strategy. The particle chooses $w$ based on $l$, which creates correlation between survival events across laps.

To get a tight upper bound, we'd need to show that no strategy can do better than some value $V^*$, then show our strategy achieves $V^*$. But characterizing $V^*$ directly seems as hard as solving the optimization problem.

\section{Conclusion}

The upper bound approach doesn't seem to yield a clean proof. The naive bounds are too loose, and tighter bounds require essentially solving the full optimization problem.

\textbf{Key insight gained:} The problem has a recursive structure --- the value of being at offset $\Omega$ depends on expected future value, which depends on the distribution of future offsets. This suggests the MDP approach (Approach 2) may be more natural.

\end{document}
