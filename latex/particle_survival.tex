\documentclass{article}

% Language setting
% Replace `english' with e.g. `spanish' to change the document language
\usepackage[english]{babel}

% Set page size and margins
% Replace `letterpaper' with `a4paper' for UK/EU standard size
\usepackage[letterpaper,top=2cm,bottom=2cm,left=3cm,right=3cm,marginparwidth=1.75cm]{geometry}

% Useful packages
\usepackage{amsmath}
\usepackage{graphicx}
\usepackage[colorlinks=true, allcolors=blue]{hyperref}

\newcommand{\numLaps}{\text{\# of laps}}

\title{Particle Survival}
\author{}
\date{}

\begin{document}
\maketitle

% \begin{abstract}
% Your abstract.
% \end{abstract}


\section{Introduction}

This document explores the survival of particles with certain parameters. In particular, the particles live in 1-dimensional space with parameters such as lifespan and velocity. A particle starts at $x=0$ and its goal is to reach $x=1$ and then come back to $x=0$ before running out of lifespan. This is referred to as lapping.

\section{Scenarios}

\subsection{Lifespan Conforming to an Exponential Distribution}

Suppose that a particle $p$ has a velocity $v>0$ and a lifespan of $l$ seconds, sampled from an exponential probability distribution with mean $1/\lambda$:
\[
P(x)=\lambda e^{-\lambda x}
\]
Then, the particle can travel from $x=0$ to $x=1$ and back to $x=0$ in $2/v$ seconds, so the probability of surviving a lap is the probability that the particle's lifespan is at least $2/v$:
\[
S(l):=\Pr(l \ge 2/v)=\int_{2/v}^{\infty}P(x)dx=e^{-2\lambda/v}.
\]
\href{https://github.com/g-jensen/stochastic-particles/blob/master/examples/survival_rate.cpp}{Simulation Code on GitHub}

\subsection{Multiple Laps}

Assume the conditions of the previous scenario, but now consider a particle doing multiple laps. In this way, $x=0$ serves as a sort of resource generator that the particle is using to stay alive while exploring. Now we can formulate some more questions:\\
\\
(1) What is the probability that a particle laps exactly $n$ times?\\
(2) What is the probability that a particle laps at least $n$ times?\\
(3) What is the expected number of laps for a particle?

\subsubsection{Basic Theory}

Regarding (1), we know that $S(l)$ is independent for each lap, so lapping exactly $n$ times is equivalent to repeatedly surviving exactly $n$ times and then not surviving once:
\[
P(\numLaps=n) = (S(l))^n(1-S(l)) = e^{-2\lambda n/v}\left(1-e^{-2\lambda/v}\right).
\]
And for (2), it's the same but without the assumption of not surviving after the $n$ resets:
\[
P(\numLaps \ge n) = (S(l))^n = e^{-2\lambda n/v}.
\]
Note that (2) is also the sum of (1) for all $k\ge n$:
\[
P(\numLaps\ge n) = \sum_{k=n}^\infty P(\numLaps = k)=\sum_{k=n}^\infty (S(l))^k(1-S(l))
\]
which, more generally, yields
\[
\sum_{k=n}^\infty (S(l))^k(1-S(l))= (S(l))^n
\]
which can be confirmed by noticing that the series is telescoping.\\
\\
Lastly, (3) can be calculated as a weighted sum of (1):
\[
\sum_{k=0}^{\infty}k\cdot P(\numLaps=k)=\sum_{k=0}^{\infty}k\cdot (S(l))^k(1-S(l))=\sum_{k=0}^{\infty}\left(k(S(l))^k-k(S(l))^{k+1}\right).
\]
We will see that this is also a telescoping sum and compute it by expanding and simplifying as follows:
\begin{align*}
& \ \sum_{k=0}^{\infty}\left(k(S(l))^k-k(S(l))^{k+1}\right)\\
= & \ (1)S(l)-(1)(S(l))^2\\
& + \ (2)(S(l))^2-(2)(S(l))^3\\
& + \ (3)(S(l))^3-(3)(S(l))^4\\
& + \ ...\\
= & \ (1)S(l) + (1)(S(l))^2 + (1)(S(l))^3 + \ ...\\
= & \ \sum_{k=1}^\infty (S(l))^n=\left(\sum_{k=0}^\infty (S(l))^n\right)-1=\frac{1}{1-S(l)}-1 =\frac{S(l)}{1-S(l)}\\
= & \frac{e^{-2\lambda/v}}{1-e^{-2\lambda/v}}=\frac{1}{e^{2\lambda/v}-1}=(e^{2\lambda/v}-1)^{-1}.
\end{align*}
It is important to note that the reasoning for the three general facts that we gathered from the questions:
\begin{align*}
    & P(\numLaps = n)=(S(l))^n(1-S(l)),\\
    & P(\numLaps \geq n)=\sum_{k=n}^\infty (S(l))^k(1-S(l))=(S(l))^n,\\
    & E(\numLaps) =\sum_{k=0}^{\infty}k\cdot (S(l))^k(1-S(l))=\frac{S(l)}{1-S(l)}
\end{align*}
did not depend on $P(x)$ being exponential. Indeed, these equations are true for any independent events with a survival probability $S(l)$.\footnote{$S(l)$ being the probability that the event continues to happen. These equations are specific to survival probability, meaning that we stop trials once the event fails to happen.}\\
\\
\href{https://github.com/g-jensen/stochastic-particles/blob/master/examples/resets.cpp}{Simulation Code on GitHub}

\subsubsection{Limit of E(\# of resets)}
An interesting fact about $E(\text{\# of resets})$ is that, with an exponential distribution, it becomes essentially linear as $v$ increases. It can be seen that $E(\text{\# of resets})$ becomes arbitrarily close to $\frac{1}{2\lambda}v - \frac{1}{2}$ if we have $S(l)=e^{-2\lambda/v}$ (as shown before) by evaluating the following limit:
\[
\lim_{v\to\infty}\left(E(\text{\# of resets}) - \frac{v}{2\lambda}\right)=\lim_{v\to\infty}\left(\frac{1}{e^{2\lambda/v}-1} - \frac{v}{2\lambda}\right)
\]
First, we can combine the fractions
\begin{align*}
\frac{1}{e^{2\lambda/v}-1} - \frac{v}{2\lambda}=\frac{2\lambda - ve^{2\lambda/v}+v}{2\lambda(e^{2\lambda/v}-1)}.
\end{align*}
Now, we can expand the numerators and denominators
\begin{align*}
    & {2\lambda - ve^{2\lambda/v}+v}=2\lambda-v\left(1+2\lambda/v+\frac{(2\lambda/v)^2}{2!}+...\right)+v=-\frac{(2\lambda)^2}{2!v}-\frac{(2\lambda)^3}{3!v^2}-\frac{(2\lambda)^4}{4!v^3}-...,\\
    & {2\lambda(e^{2\lambda/v}-1)}=2\lambda e^{2\lambda/v}-2\lambda=2\lambda\left(1+2\lambda/v+\frac{(2\lambda/v)^2}{2!}+...\right)-2\lambda=\frac{(2\lambda)^2}{v}+\frac{(2\lambda)^3}{2!v^2}+\frac{(2\lambda)^4}{3!v^3}+...\\
    & \implies \frac{2\lambda - ve^{2\lambda/v}+v}{2\lambda(e^{2\lambda/v}-1)}=\frac{-\frac{(2\lambda)^2}{2!v}-\frac{(2\lambda)^3}{3!v^2}-\frac{(2\lambda)^4}{4!v^3}-...}{\frac{(2\lambda)^2}{v}+\frac{(2\lambda)^3}{2!v^2}+\frac{(2\lambda)^4}{3!v^3}+...}=\frac{\frac{(2\lambda)^2}{v}\left(-\frac{1}{2!}-\frac{2\lambda}{3!v}-\frac{(2\lambda)^2}{4!v^2}-...\right)}{\frac{(2\lambda)^2}{v}\left(1+\frac{2\lambda}{2!v}+\frac{(2\lambda)^2}{3!v^2}+...\right)}
\end{align*}
and then look at the limit
\begin{align*}
\lim_{v\to\infty}\left(\frac{1}{e^{2\lambda/v}-1} - \frac{v}{2\lambda}\right)=\lim_{v\to\infty}\frac{-\frac{1}{2!}-\frac{2\lambda}{3!v}-\frac{(2\lambda)^2}{4!v^2}-...}{1+\frac{2\lambda}{2!v}+\frac{(2\lambda)^2}{3!v^2}+...}=-\frac{1}{2}.
\end{align*}
Thus,
\[
\lim_{v\to\infty}\left(\frac{1}{e^{2\lambda/v}-1} - \left(\frac{v}{2\lambda}-\frac{1}{2}\right)\right)=0
\]
so $E(\text{\# of resets})$ becomes arbitrarily close to the line $\frac{1}{2\lambda}v-\frac{1}{2}$ as $v$ approaches infinity.

\subsection{Gates and Wait Times}
Now imagine $x=0$ as a sort of gate that turns on and off periodically. If the particle reaches $x=0$ when the gate is on, then the particle is allowed to reset and keep going. If the particle reaches $x=0$ when the gate is off, it dies. Specifically, the gate is on for $[0,T_1)$ and then off for $[T_1,T_1+T_2)$ and repeats this pattern indefinitely. Based on its velocity, the particle may unluckily become out-of-sync with the gate and frequently die, so we give it the ability to wait. Waiting still drains the life of the particle, but it can wait for an amount of time to try make sure that it reaches $x=0$ when the gates is on. The particle can wait different times based on the lifespan it was given after each lap.
\\
\\
The question now becomes: what is the optimal wait time during a given lap? Optimal meaning that the strategy that yields the highest expected number of resets.
\\
\\
The particle can wait at any period during the lap, but it is equivalent if it only waits at the beginning of a lap.
\\
\\
By simplifying to a scenario starting at $t=0$, I've calculated that if the gate starts on, then
\[
\Delta t = \begin{cases}
  l - \frac{2}{v}, & l \mod T_1+T_2 \leq T_1\\
  T_1+\lfloor \frac{l}{T_1+T_2}\rfloor(T_1+T_2)-\frac{2}{v}, & l \mod T_1+T_2 > T_1
\end{cases}
\]
is the furthest point that the particle can start from and still live. Since every lap starts at some offset during an interval when the gate is on, say $\Omega$. Then we have the following equation regard the wait time during a lap:
\[
w + \Omega = \Delta t \implies w= \Delta t - \Omega.
\]
\bibliographystyle{alpha}
\bibliography{sample}

\end{document}
